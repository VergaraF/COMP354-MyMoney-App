%COMP 354 Project
%Iteration III Documentation: TESTING

\documentclass{article}
\usepackage[utf8]{inputenc}
\usepackage{fancyhdr}
%Packages
\usepackage[utf8]{inputenc} %list of tables
\usepackage{array}
\usepackage{tabu}
\usepackage{multirow}
\usepackage{tikz}
\usepackage{graphicx}
\usepackage{float}
\pagestyle{fancy} % options: empty , plain , fancy
\renewcommand{\headrulewidth}{0pt} % customize the layout...
\lhead{}\chead{}\rhead{}

\title{Project: MyMoney App \newline Iteration 3}

\author{ }
\date{April 2018}
\rhead{MyMoneyApp -version 1.2}
\rfoot{COMP354 - PA-PJ-Winter 2018}


\begin{document}

\maketitle

\vspace*{1.5in}
\begin{table}[htbp]

\begin{center}
\begin{tabular}{|l | l|}
\hline
Name & ID Number \\
\hline
Fabian Vergara & 40006707 \\
\hline

Jean-Loup  & 40006259 \\
\hline
Vincent Boivin & 27419694 \\

\hline
Marc-Antoine Jette-Leger & 27895038 \\
\hline
Michael Xu & 27206356 \\
\hline
Kisife Giles & 40001926 \\
\hline
\end{tabular}
\end{center}
\caption{Team: PA-PJ}
\label{table:1}
\end{table}


\newpage
\tableofcontents

\newpage

\section{Introduction}
This document marks the end of the third and last phase of the three iterations in the COMP 354 group project. In the course of these three iterations, team PA-PJ worked on building the MyMoneyApp which is simple application to provide financial advice to young professionals.

\subsection{Purpose}
The objective in this phase of the project is to test the application against the requirements ( for validity), robustness, and reliability.
To this end, a test plan has been made. The plan presents the test categories used, the components, units or subsystems to be tested and the approach used. However, due to time constrains, not all components could be tested. The tested components are presented in section3, while those excluded are indicated in section 4.




\newpage

\section{Outline of Planned Tests}
The following tests will be carried out on the MyMoneyApp. Due to time constrains, we will do only the black box testing and exclude white box testing. This means that the test carried out will test for inputs and outputs from the system, but not the code coverage.

\subsubsection{Unit Testing}
Unit testing tests individual units in the system.Here we test the core functionalists of individual classes for their behaviours. Black box test were carried out on the following methods of the following classes:
\begin{itemize}
\item RegistrationTest Class, for user registration.
    \item LogIn Class, for user login activity.
    \item Finance Class, for finance activities.
    \item Advice Class for providing advise to the user.
    \item BankingInfo Class for bank transactions.
    \item Display Class for accuracy of information displayed to the user.
    \item Encryption Class for encryption of user information.
    
\end{itemize}
The details of the tested units is presented in section 5.1 below.


\subsubsection{Integration Testing}
Involves testing how units work together to achieve tasks and responsibilities. Integration testing will be done on the following collaborating classes/units:

\begin{itemize}
    \item LoginTest Class and BankingInfo Class for collaboration to login and upload the correct user information.
    \item Registration and Encryption classes.
    \item Advice class and Display class  
\end{itemize}


\clearpage

\subsubsection{Functional Testing}
This test's purpose is to ensure the functionality of the software system against any type of input. It is a type of black-box testing, which is a type of testing that disregards the internal program structure and focuses on what the system does, rather than how it is made. To this end, what the system does is first identified then a test to check its functionality and reliability is designed based on the expected behaviour, then fault tolerance(robustness) in the case of invalid input.
Functional testing has been run on these requirements:
\begin{itemize}
    \item Registration and Login
    \item Data View
    \item Set Goals
    \item Data Input Automation
    \item Advice
    \item Spending History View
    \item Dynamic Screen Adjustment
    \item Savings Feature
\end{itemize}

\clearpage

\subsubsection{User Interface Testing}
User interface testing is a testing method that involves finding  defects in the software system through the use of the user interface to in turn, ensure that it meets the System Requirements Specifications.
\break
The application provides the following user interfaces which where tested and documented:


\begin{itemize}
    \item The Main Panel
    \item Registration Panel
    \item Bank Panel
    \item Budgeting Panel
    \item Results Chart
\end{itemize}

\newpage

\section{Test Inclusions}
This section specifies the core tests done on the application.
\subsection{Unit Test}

Unit testing is done with black box only, due to time constraints.

\subsection{Integration Testing}
The individually tested units are combined and tested to evaluate how they work together.
\subsection{Functional Testing}
Test cases are generated to match individual use cases.

\subsection{User Interface Testing}
Here the GUI is tested against user interaction.

\subsection{Security and Access Control Testing}
The application is tested against unauthorized access.

\subsection{Load Testing}
The application is tested against large input files.

\section{Test Exclusions}

The following test cases were deemed necessary for the MyMonetApp but will be excluded due to time constraints. White box/ code coverage test is excluded from the unit tests.


\subsection{Data and Database Integrity Testing}
In this project, simple text files are used in place of a database. Ideally, the application would use a database. In either case, this test would be necessary.

\newpage



\section{Testing Approach}
\subsection{Unit testing}
\\AdviceTest Class : Test Case 1 : It tests to see if the input is smaller than 4 characters
\\Input : ""
\\Expected Output : Error
\\Output : Error
\\Bug : No
\\\\AdviceTest Class : Test Case 2 : It tests to see if the input is an advice object
\\Input : Advice object ("-230960.00 2016-02-18 Student Loans Debt")
\\Expected Output : ""
\\Output : ""
\\Bug : No
\\\\AdviceTest Class : Test Case 3 : It tests to see if the output is correct
\\Input : Advice object("-230960.00 2016-02-18 Student Loans Debt")
\\Expected Output : "You should be paying your debts rather than spending, as you have -230960.0 in debt and 0.0 in savingsHowever, you are accumulating money, as your bring in more than you use"
\\Output : "You should be paying your debts rather than spending, as you have -230960.0in debt and 0.0 in savings. However, you are accumulating money, as your bring in more than you use"
\\Bug : No
\\\\BankingInfoTest Class : Test Case 1 : It test that the transaction file is opened and not empty
\\Input : ""
\\Expected Output : true
\\Output : true
\\Bug : No
\\\\BankingInfoTest Class : Test Case 2 : It tests that the bank file is opened and not empty
\\Input : ""
\\Expected Output : true
\\Output : true
\\Bug : No
\\\\BankingInfoTest Class : Test Case 3 : It tests to see that the transactions are sorted by dates
\\Input : Transactions
\\Expected Output : true
\\Output : true
\\Bug : No
\\\\BankingInfoTest Class : Test Case 4 : It tests to see if the sorted files are created
\\Input : Created files
\\Expected Output : true
\\Output : true
\\Bug : No
\\\\BankingInfoTest Class : Test Case 5 : It tests to see if the files are saved correctly
\\Input : Created files
\\Expected Output : true
\\Output : true
\\Bug : No
\\\\DisplayTest Class : Test Case 1 : It tests that the panel for the BankPanel class is of size 960x720
\\Input : A BankPanel object
\\Expected Output : true
\\Output : true
\\Bug : No
\\\\DisplayTest Class : Test Case 2 : It tests that the panel for the FinancePanel class is of size 960x720
\\Input : A FinancePanel object
\\Expected Output : true
\\Output : true
\\Bug : No
\\\\DisplayTest Class : Test Case 3 : It tests that the panel for the log in panel is of size 960x720
\\Input : A MainPanel object
\\Expected Output : true
\\Output : true
\\Bug : No
\\\\DisplayTest Class : Test Case 4 : It test that the panel for the registration panel is of size 960x720
\\Input : A RegistrationPanel object
\\Expected Output : true
\\Output : true
\\Bug : No
\\\\EncryptionTest Class : Test Case 1 : It tests that a string is properly encrypted
\\Input : Unencrypted string
\\Expected Output : Encrypted string
\\Output : Encrypted string
\\Bug : No
\\\\EncryptionTest Class : Test Case 2 : It tests that a string is properly encrypted and decrypted
\\Input : Encrypted string
\\Expected Output : Decrypted string
\\Output : Decrypted string
\\Bug : No
\\\\FinanceTest Class : Test Case 1 : It tests that the value homeBudget is set to 1
\\Input : homeBudget
\\Expected Output : 1
\\Output : 0
\\Bug : Yes
\\\\FinanceTest Class : Test Case 2 : It tests that the value homeSpending is set to 0
\\Input : homeSpending
\\Expected Output : 0
\\Output : 0
\\Bug : No
\\\\FinanceTest Class : Test Case 3 : It tests that the value of text1 is set to ""
\\Input : text1
\\Expected Output : ""
\\Output : ""
\\Bug : No
\\\\LogInTest Class : Test Case 1 : It tests for the size of the list containing user information
\\Input : Size of the list
\\Expected Output : Size of the list
\\Output : Size of the list
\\Bug : No
\\\\LogInTest Class : Test Case 2 : It tests that the size of the list is greater than 0
\\Input : Size of the list
\\Expected Output : true
\\Output : true
\\Bug : No
\\\\LogInTest Class : Test Case 3 : It tests to see if the input is contained in the user information list
\\Input : test data
\\Expected Output : "Failure validating the user information!"
\\Output : "Failure validating the user information!"
\\Bug : No

\break

\subsection{Integration Testing}




\begin{table}[htbp]
%\caption{Team: PA-PJ}
\begin{center}
\begin{tabular}{|l | l|}
\hline
Test Attribute & Test Specifications \\
\hline
Test ID & INTG-1 \\
\hline
Test Name  & Login and BankInfo Integration \\
\hline
Test Suite  ID & INTG \\
\hline
Test preconditions & An existing user account and bank transactions \\
\hline
Test Steps & Case 1: Log in to user's account \\

\hline
Test post conditions & --- \\
\hline
Test Case & Expected Results\\
\hline
Case 1  &  Bank transactions uploaded should belong to corresponding user account  \\
\hline
  Case 2 &  Registration panel should still be displayed  \\
\hline
Actual Results & ---\\
\hline
Case 1  & Corresponding bank transactions were loaded.  \\
\hline
Test Results & ---\\
\hline
Failed Tests & None\\
\hline
Passed Tests & Case 1 \\
\hline
\end{tabular}
\end{center}
\caption{Integration Test 1}
\end{table}
\label{table:2a -intg}



\begin{table}[htbp]
%\caption{Team: PA-PJ}
\begin{center}
\begin{tabular}{|l | l|}
\hline
Test Attribute & Test Specifications \\
\hline
Test ID & INTG-1 \\
\hline
Test Name  & Registration and Encryption \\
\hline
Test Suite  ID & INTG \\
\hline
Test preconditions & Username and password should not match an existing account in the system. \\
\hline
Test Steps & Case 1: Create a new account and upload bank transactions. \\

\hline
Test post conditions & --- \\
\hline
Test Case & Expected Results\\
\hline
Case 1  &  User information and bank transactions should be encrypted after registration. \\

\hline
Actual Results & ---\\
\hline
Case 1  & User login information and bank transactions were encrypted.  \\
\hline
Test Results & ---\\
\hline
Failed Tests & None\\
\hline
Passed Tests & Case 1 \\
\hline
\end{tabular}
\end{center}
\caption{Integration Test 2}
\end{table}
\label{table:2a -intg2}

\clearpage

\subsection{Functional Testing}

These tests aim to ensure the functionality and presence of the requirements that have been put forth in the SRS document.

\subsubsection{Registration Functional tests}
\begin{table}[htbp]
%\caption{Team: PA-PJ}
\begin{center}
\begin{tabular}{|l | l|}
\hline
Test Attribute & Test Specifications \\
\hline
Test ID & FUNC-1 \\
\hline
Test Name  & Username should be unique \\
\hline
Test Suite  ID & FUNC \\
\hline
Test preconditions & Username must not exist beforehand \\
\hline
Test Steps & Case 1: Click on registration \\
\hline
  & Case 2: Input username that exists already  \\
\hline
  & Case 3: Input a password  \\
\hline
  & Case 4: Click register account button  \\
\hline
Test post conditions & --- \\
\hline
Test Case & Expected Results\\
\hline
Case 1  &  Invalid username warning should appear  \\
\hline
  Case 2 &  Registration panel should still be displayed  \\
\hline
Actual Results & ---\\
\hline
Case 1  &  Invalid username warning appears  \\
\hline
Case 2 &  Registration panel still be displayed  \\
\hline
Test Results & ---\\
\hline
Failed Tests & None\\
\hline
Passed Tests & Case 1 case 2\\
\hline
\end{tabular}
\end{center}
\caption{Registration functional test 1}
\end{table}
\label{table:2a}


\begin{table}[htbp]
%\caption{Team: PA-PJ}
\begin{center}
\begin{tabular}{|l | l|}
\hline
Test Attribute & Test Specifications \\
\hline
Test ID & FUNC-2 \\
\hline
Test Name  & Username, Password are mandatory \\
\hline
Test Suite  ID & FUNC \\
\hline
Test preconditions & The user cannot sign up without providing a username and a password \\
\hline
Test Steps & Case 1: Click on registration \\
\hline
  & Case 2: Leave everything blank  \\
\hline
  & Case 3: click on Register Account \\
\hline
Test post conditions & --- \\
\hline
Test Case & Expected Results\\
\hline
Case 1  &  Please enter username, password warning should appear  \\
\hline
Case 2 &  Registration panel should still be displayed  \\
\hline
Actual Results & ---\\
\hline
Case 1  &  please enter username,password warning appears  \\
\hline
  Case 2 &  Registration panel still displayed  \\
\hline
Test Results & ---\\
\hline
Failed Tests & None\\
\hline
Passed Tests & Case 1 case 2\\
\hline
\end{tabular}
\end{center}
\caption{Registration Functional Test 2}
\end{table}
\label{table:2b}


\begin{table}[htbp]
%\caption{Team: PA-PJ}
\begin{center}
\begin{tabular}{|l | l|}
\hline
Test Attribute & Test Specifications \\
\hline
Test ID & FUNC-3 \\
\hline
Test Name  & Account is successfully created \\
\hline
Test Suite  ID & FUNC \\
\hline
Test preconditions & --- \\
\hline
Test Steps & Case 1: Click on registration \\
\hline
  & Case 2: Enter an unused username  \\
\hline
  & Case 3: click on Register Account \\
\hline
Test post conditions & --- \\
\hline
Test Case & Expected Results\\
\hline
Case 1  &  Registration page should close\\
\hline
Case 2 &  User is back to the login page  \\
\hline
Actual Results & ---\\
\hline
Case 1  &  Registration page closes\\
\hline
Case 2 &  User is back to the login page  \\
\hline
Test Results & ---\\
\hline
Failed Tests & None\\
\hline
Passed Tests & Case 1 case 2\\
\hline
\end{tabular}
\end{center}
\caption{Registration Functional Test 3}
\end{table}
\label{table:2c}


\newpage

\subsubsection{Login Functional tests}

\begin{table}[htbp]
%\caption{Team: PA-PJ}
\begin{center}
\begin{tabular}{|l | l|}
\hline
Test Attribute & Test Specifications \\
\hline
Test ID & FUNC-4 \\
\hline
Test Name  & User logs in properly \\
\hline
Test Suite  ID & FUNC \\
\hline
Test preconditions & User has an account registered in the system \\
\hline
Test Steps & Case 1**: Input username and password in the appropriate boxes \\
\hline
  & Case 2**: Click on the login button \\
\hline
Test post conditions & --- \\
\hline
Test Case**: & Expected Results\\
\hline
Case 1**:  &  Login page should disappear\\
\hline
Case 2: &  User is at the main bank panel  \\
\hline
Actual Results & ---\\
\hline
Case 1  &  Login page disappears\\
\hline
Case 2 &  User is at the bank panel page  \\
\hline
Test Results & ---\\
\hline
Failed Tests & None\\
\hline
Passed Tests & Case 1, Case 2\\
\hline
\end{tabular}
\end{center}
\caption{Login Functional Test 1}
Content\footnote{** Authentication test may be applicable in section 5.}
\end{table}
\label{table:2d}


\begin{table}[htbp]
%\caption{Team: PA-PJ}
\begin{center}
\begin{tabular}{|l | l|}
\hline
Test Attribute & Test Specifications \\
\hline
Test ID & FUNC-5 \\
\hline
Test Name  & Username/Password pair invalid/nonexistent \\
\hline
Test Suite  ID & FUNC \\
\hline
Test preconditions & --- \\
\hline
Test Steps & Case 1: Input username and password in the appropriate boxes \\
\hline
  & Case 2: Click on the login button \\
\hline
Test post conditions & --- \\
\hline
Test Case & Expected Results\\
\hline
Case 1**:  &  Invalid username/password should appear\\
\hline
Case 2**: &  User is still at the login page  \\
\hline
Actual Results & ---\\
\hline
Case 1**:  &  invalid username/password appears\\
\hline
Case 2 &  User is at the login page  \\
\hline
Test Results & ---\\
\hline
Failed Tests & None\\
\hline
Passed Tests & Case 1, Case 2\\
\hline
\end{tabular}
\end{center}
\caption{Login Functional Test 2}
Content\footnote{** Authentication test may be applicable in section 5.}
\end{table}
\label{table:2e}

\clearpage

\subsubsection{Data View}

\begin{table}[htbp]
%\caption{Team: PA-PJ}
\begin{center}
\begin{tabular}{|l | l|}
\hline
Test Attribute & Test Specifications \\
\hline
Test ID & FUNC-6 \\
\hline
Test Name  & Data view \\
\hline
Test Suite  ID & FUNC \\
\hline
Test preconditions & User must have a valid pair of username/password \\
\hline
Test Steps & Case 1: Input username and password in the appropriate boxes \\
\hline
  & Case 2: Click on the login button \\
\hline
Test post conditions & User is able to view his financial data \\
\hline
Test Case & Expected Results\\
\hline
Case 1  &  User logs in successfully\\
\hline
Case 2 &  User is at the main panel, with the data ready to be viewed  \\
\hline
Actual Results & ---\\
\hline
Case 1  &  User logs in successfully\\
\hline
Case 2 &  User is at the main panel, with the data ready to be viewed  \\
\hline
Test Results & ---\\
\hline
Failed Tests & None\\
\hline
Passed Tests & Case 1, Case 2\\
\hline
\end{tabular}
\end{center}
\caption{Transactions View Functional Test}
\end{table}
\label{table:2f}

\clearpage

\subsubsection{Set Goals}

\begin{table}[htbp]
%\caption{Team: PA-PJ}
\begin{center}
\begin{tabular}{|l | l|}
\hline
Test Attribute & Test Specifications \\
\hline
Test ID & FUNC-7 \\
\hline
Test Name  & Data view \\
\hline
Test Suite  ID & FUNC \\
\hline
Test preconditions & User has logged into a pre-existing account and is now on the main bank panel \\
\hline
Test Steps & Case 1: Click on budget\\
\hline
  & Case 2: Input the amount for each category \\
 \hline
  & Case 3: Click on add to budget \\
\hline
Test post conditions & User should see his budget increased and can have access to a chart \\
\hline
Test Case & Expected Results\\
\hline
Case 1  &  User gets to the budget page\\
\hline
Case 2 &  User inputs the required information  \\
\hline
Case 3 &  User has access to a chart of his spending  \\
\hline
Actual Results & ---\\
\hline
Case 1  &  User gets to the budget page\\
\hline
Case 2 &  User inputs the required information  \\
\hline
Case 3 &  User has access to a chart of his spendings  \\
\hline
Test Results & ---\\
\hline
Failed Tests & None\\
\hline
Passed Tests & Case 1, Case 2,Case 3\\
\hline
\end{tabular}
\end{center}
\caption{Set Goals Functional Test}
\end{table}
\label{table:2g}

\clearpage

\subsubsection{Data Input Automation}

\begin{table}[htbp]
%\caption{Team: PA-PJ}
\begin{center}
\begin{tabular}{|l | l|}
\hline
Test Attribute & Test Specifications \\
\hline
Test ID & FUNC-8 \\
\hline
Test Name  & Data view \\
\hline
Test Suite  ID & FUNC \\
\hline
Test preconditions & User has logged into a pre-existing account and is now on the main bank panel \\
\hline
Test Steps & Case 1: The user's financial spending is displayed\\
\hline
Test post conditions & User is able to see his spending \\
\hline
Test Case & Expected Results\\
\hline
Case 1  &  User is able to see his spending\\
\hline
Actual Results & ---\\
\hline
Case 1  &  User is able to see his spending\\
\hline
Test Results & ---\\
\hline
Failed Tests & None\\
\hline
Passed Tests & Case 1\\
\hline
\end{tabular}
\end{center}
\caption{Data Input Automation Functional Test}
\end{table}
\label{table:2h}

\clearpage

\subsubsection{Advice Generation}

\begin{table}[htbp]
%\caption{Team: PA-PJ}
\begin{center}
\begin{tabular}{|l | l|}
\hline
Test Attribute & Test Specifications \\
\hline
Test ID & FUNC-9 \\
\hline
Test Name  & Advice generation based on input \\
\hline
Test Suite  ID & FUNC \\
\hline
Test preconditions & User has logged into a pre-existing account and is now on the main bank panel \\
\hline
Test Steps & Case 1: The user's advice on his spending is displayed\\
\hline
Test post conditions & User is able to see his advice \\
\hline
Test Case & Expected Results\\
\hline
Case 1  &  User is able to see his advice\\
\hline
Actual Results & ---\\
\hline
Case 1  &  User is able to see his advice\\
\hline
Test Results & ---\\
\hline
Failed Tests & None\\
\hline
Passed Tests & Case 1\\
\hline
\end{tabular}
\end{center}
\caption{Advice Generation Functional Test}
\end{table}
\label{table:2i}

\clearpage

\subsubsection{Spending History View}

\begin{table}[htbp]
%\caption{Team: PA-PJ}
\begin{center}
\begin{tabular}{|l | l|}
\hline
Test Attribute & Test Specifications \\
\hline
Test ID & FUNC-10 \\
\hline
Test Name  & Able to sort the spending based on time \\
\hline
Test Suite  ID & FUNC \\
\hline
Test preconditions & User has logged into a pre-existing account and is now on the main bank panel \\
\hline
Test Steps & Case 1: Click on date\\
\hline
Test post conditions & The spending history has been sorted by date \\
\hline
Test Case & Expected Results\\
\hline
Case 1  &  Spending history is sorted by date\\
\hline
Actual Results & ---\\
\hline
Case 1  &  Spending history is sorted by date\\
\hline
Test Results & ---\\
\hline
Failed Tests & None\\
\hline
Passed Tests & Case 1\\
\hline
\end{tabular}
\end{center}
\caption{Spending History View Functional Test}
\end{table}
\label{table:2j}

\clearpage

\subsubsection{Dynamic Screen Adjustment Test}
This test case tests the use case 5; the application should dynamically adjust the screen settings to suit the host device settings.

\begin{table}[htbp]
%\caption{Team: PA-PJ}
\begin{center}
\begin{tabular}{|l | l|}
\hline
Test Attribute & Test Specifications \\
\hline
Test ID & FUNC-10 \\
\hline
Test Name  & Dynamic change of screen size \\
\hline
Test Suite  ID & FUNC \\
\hline
Test preconditions & User has logged into a pre-existing account and is now on the main bank panel \\
\hline
Test Steps & Case 1: Click on date\\
\hline
Test post conditions & The screen size adjusts to fit the information \\
\hline
Test Case & Expected Results\\
\hline
Case 1  &  The screen size adjusts to fit the information\\
\hline
Actual Results & ---\\
\hline
Case 1  &  The screen size adjusts to fit the information\\
\hline
Test Results & ---\\
\hline
Failed Tests & None\\
\hline
Passed Tests & Case 1\\
\hline
\end{tabular}
\end{center}
\caption{Dynamic Screen Adjustment Functional Test}
\end{table}
\label{table:2k}

\clearpage

\subsubsection{Savings Advice Functional Test}

\begin{table}[htbp]
%\caption{Team: PA-PJ}
\begin{center}
\begin{tabular}{|l | l|}
\hline
Test Attribute & Test Specifications \\
\hline
Test ID & FUNC-10 \\
\hline
Test Name  & Saving information is given(separate from regular advice) \\
\hline
Test Suite  ID & FUNC \\
\hline
Test preconditions & User has logged into a pre-existing account and is now on the main bank panel \\
\hline
Test Steps & Case 1: Saving advice is displayed in the middle column\\
\hline
Test post conditions &  Saving advice is displayed in the middle column \\
\hline
Test Case & Expected Results\\
\hline
Case 1  &  Saving advice is displayed in the middle column\\
\hline
Actual Results & ---\\
\hline
Case 1  &  Saving advice is displayed in the middle column\\
\hline
Test Results & ---\\
\hline
Failed Tests & None\\
\hline
Passed Tests & Case 1\\
\hline
\end{tabular}
\end{center}
\caption{Savings Advice Functional Test}
\end{table}
\label{table:2l}

\break



\subsection{User Interface Testing}

Each user interface was tested for functionality against user interactions. The test for each interface is presented in tabular form below.

\subsubsection{Main panel interface Test}
\begin{table}[htbp]
%\caption{Team: PA-PJ}
\begin{center}
\begin{tabular}{|l | l|}
\hline
Test Attribute & Test Specifications \\
\hline
Test ID & GUI-1 \\
\hline

Test Name  & Main Panel Test \\
\hline
Test Suite  ID& GUI \\

\hline
Test preconditions & Panel must be displayed \\
\hline
Test Steps & Case 1*: Click to close panel \\
\hline
Case 2*  &  Click panel window to maximize  \\
\hline
Case 3*  &  Click panel window to minimize  \\
\hline
  Case 4 &  Registration panel should be displayed  \\
\hline
  Case 5 &  Login activity should be initiated \\
\hline
 Case 6 &  Return to main panel  \\
\hline
Test post conditions & --- \\
\hline
Test Case & Expected Results\\
\hline
\hline
Case 1*  &  Panel should close  \\
\hline
Case 2*  &  Panel window can be maximized  \\
\hline
Case 3*  &  Panel window can be minimized  \\
\hline
  Case 4 &  Registration panel should be displayed  \\
\hline
  Case 5 &  Login activity should be initiated \\
\hline
 Case 6 &  Return to main panel  \\
\hline

Actual Results & ---\\
\hline
Case 1*  &  Panel window closed  \\
\hline
Case 2*  &  Panel window maximized  \\
\hline
Case 3*  &  Panel window minimized  \\
\hline
  Case 4 &  Registration panel displayed  \\
\hline
  Case 5 &  Login activity initiated  \\
\hline
 Case 6 &  Return to main panel  \\
\hline

Test Results & ---\\
\hline
Failed Tests & None\\
\hline
Passed Tests & Case 1 case 2 case 3 case 4 case 5 case 6\\
\hline

\end{tabular}
\end{center}
\caption{GUI Test: Main Panel Interface}
\end{table}
Content\footnote{* Signifies tests common to all panels}
\label{table:2m}

\newpage

\subsubsection{Registration Interface Test}


\begin{table}[htbp]
%\caption{Team: PA-PJ}
\begin{center}
\begin{tabular}{|l | l|}
\hline
Test Attribute & Test Specifications \\
\hline
Test ID & GUI-2 \\
\hline
Tester  & Coder 1 \\
\hline

Test Name  & Registration Panel Test \\
\hline
Test Suite  ID& GUI \\

\hline
Test preconditions & Panel must be displayed \\
\hline
Test Steps & Case 1: Click to close panel \\
\hline
  & Case 2: User name field  \\
\hline
  & Case 3: Confirm password field  \\
\hline
  & Case 4: Registration button  \\
\hline
& Case 5: Click cancel button  \\
\hline

Test post conditions & --- \\
\hline
Test Case & Expected Results\\
\hline
\hline
Case 1  &  Panel should close  \\
\hline
  Case 2 &  User name field should be displayed  \\
\hline
  Case 3 & Password field should be displayed  \\
\hline
 Case 4 & Confirm password field should be displayed  \\
\hline
Case 5 & Registration activity  \\
\hline

Actual Results & ---\\
\hline
\hline
Case 1  &  Panel close as expected  \\
\hline
  Case 2 &  User name field displayed  \\
\hline
  Case 3 & Password field displayed  \\
\hline
 Case 4 & Confirm password field displayed  \\
\hline
Case 5 & Registration activity initiated  \\
\hline
Test Results & ---\\
\hline
Failed Tests & None\\
\hline
Passed Tests & Case 1 case 2 case 3 case 4 case 5 case 6\\
\hline
Remarks & A remark if necessary \\
\hline

\end{tabular}
\end{center}

\caption{GUI Test: Registration Panel Interface}
\end{table}
Content\footnote{* Cases in main panel interface tested.}
\label{table:3xxx}

\newpage

\subsubsection{Bank Panel Interface Test}

\begin{table}[htbp]
%\caption{Team: PA-PJ}
\begin{center}
\begin{tabular}{|l | l|}
\hline
Test Attribute & Test Specifications \\
\hline
Test ID & GUI-3 \\
\hline
Tester  & Coder 2 \\
\hline

Test Name  & Bank Panel Test \\
\hline
Test Suite  ID& GUI \\

\hline
Test preconditions & Panel must be displayed \\
\hline
Test Steps & Case 1: Advice text area should displayed \\
\hline
  & Case 2: Savings text area should displayed  \\
\hline
  & Case 3: Transactions text area should displayed  \\
\hline
  & Case 4: Click on Raw button  \\
\hline
& Case 5: Click date button  \\
\hline
& Case 6: Click exit button  \\
\hline
& Case 7: Click budget button  \\
\hline

Test post conditions & --- \\
\hline
Test Case & Expected Results\\
\hline
Case 1 &     Advice should be displayed in advice text area  \\
\hline
Case 2 &  Savings should be displayed on savings area should displayed  \\
\hline
Case 3  &  Transactions should be displayed on text area should displayed  \\
\hline
Case 4  &  Listener event should be initiated for raw data  \\
\hline
Case 5 &  Listener event should be initiated and date displayed  \\
\hline
Case 6 &  Application should exit  \\
\hline
Case 7 &  User should be taken to budget panel  \\
\hline

Actual Results & ---\\
\hline
Case 1 & Advice is displayed in advice text area  \\
 \hline
Case 2 & Savings are displayed on savings area should displayed  \\
\hline
Case 3 &  Transactions are displayed on text area should displayed  \\
\hline
 Case 4 &  raw date data is presented \\
\hline
Case 5 &  Transactions are displayed by date  \\
\hline
Case 6 &  Application terminates \\
\hline
Case 7 &  User is taken to budget panel  \\
\hline
Test Results & ---\\
\hline
Failed Tests & None\\
\hline
Passed Tests & Case 1 case 2 case 3 case 4 case 5 case 6 case 7\\
\hline


\end{tabular}
\end{center}

\caption{GUI Test: Bank Panel Interface}
Content\footnote{* Cases in main panel interface tested.}
\end{table}
\label{table:3xxxxx}


\clearpage

\subsubsection{Budget Panel Interface Test}
This interface should enable a user to set a budget.


\begin{table}[htbp]
%\caption{Team: PA-PJ}
\begin{center}
\begin{tabular}{|l | l|}
\hline
Test Attribute & Test Specifications \\
\hline
Test ID & GUI-4 \\
\hline
Tester  & Coder 1 \\
\hline

Test Name  & Budget Panel Test \\
\hline
Test Suite  ID& GUI \\

\hline
 Test preconditions & Panel must be displayed \\
\hline
Test Steps & Case 1: Budget text area should displayed \\
\hline
  & Case 2: Categories drop-down window should displayed  \\
\hline
  & Case 3: Budget entry fields should displayed  \\
\hline
  & Case 4: Click on "Set" button  \\
\hline
& Case 5: Click "Cancel" button  \\
\hline
& Case 6: Click exit button  \\
\hline
& Case 7: Click the pie chart button\\
\hline

Test post conditions & --- \\
\hline
Test Case &Expected Results\\
\hline
Case 1 & Budget setting area should be displayed in advice text area  \\
\hline
Case 2 &  Categories drop-down window should be displayed  \\
\hline
Case 3  &  Budget entry fields should be displayed  \\
\hline
Case 4  &  Listener event should be initiated for budget activity  \\
\hline
Case 5 &  Listener event should be initiated and user returned to main panel \\
\hline
Case 6 &  Application should exit  \\
\hline
Case 7 &  User should be taken to pie chart panel  \\
\hline

Actual Results & ---\\
\hline
Case 1 & Budget area displayed as expected  \\
 \hline
Case 2 & Budget categories presented as expected  \\
\hline
Case 3 &  Budget entry fields displayed  \\
\hline
 Case 4 &  Budget activity started \\
\hline
Case 5 &  Transactions are displayed by date  \\
\hline
Case 6 &  Application terminates \\
\hline
Case 7 &  User is taken to pie chart panel  \\
\hline
Test Results & ---\\
\hline
Failed Tests & None\\
\hline
Passed Tests & Case 1 case 2 case 3 case 4 case 5 case 6 case 7\\
\hline


\end{tabular}
\end{center}

\caption{GUI Test: Budget Panel Interface}
Content\footnote{* Cases in main panel interface apply.}
\end{table}
\label{table:}




\clearpage


\subsection{Security and Access Control Testing}

\begin{table}[htbp]
%\caption{Team: PA-PJ}
\begin{center}
\begin{tabular}{|l | l|}
\hline
Test Attribute & Test Specifications \\
\hline
Test ID & SEC-1 \\
\hline
Test Name  & Security Test \\
\hline
Test Suite  ID& Sec \\
\hline
 Test preconditions & --- \\
\hline
Test Steps & Case 1: Click on register \\
\hline
Test Steps & Case 2: Input username/password \\
\hline
Test post conditions & --- \\
\hline
Test Case & Expected Results\\
\hline
Case 1 & All data is encrypted  \\
\hline
Actual Results & ---\\
\hline
Case 1 & Data is encrypted  datafile \textbackslash login\_info \\
\hline
Test Results & ---\\
\hline
Failed Tests & None\\
\hline
Passed Tests & Case 1 \\
\hline
\end{tabular}
\end{center}

\caption{Security and access Test}
Content\footnote{* Cases in main panel interface apply.}
\end{table}
\label{table:}

\newpage





\subsection{Load Testing}

Loading large files fails in that the Bank Bank Panel GUI can not accommodate all of the information. 
\newline MyMoneyApp algorithm handles different loads of different magnitudes but fails to display everything correctly. This test puts in evidence a cosmetic flaw of the system that needs to be addressed.

Remark: A scrollable GUI could solve this problem.





\section{Description of Input Files}

MyMoneyApp receives data stream from bank institutions in the form of text files with extension .txt. There is one text file for every bank data stream that will be provided from MyMoneyApp users' banks. However, for this iteration and to run most of the tests, the development team only uses one input file. \newline 
This input file is extremely important as most of MyMoneyApp functionality depends on such. Without this file, no useful information is displayed to a given user, and the user is informed that a data file is needed to be provided in order to enjoy MyMoneyApp features. \newline\newline

\textbf{Transactions\textunderscore data.txt} corresponds to the input file name and its respective extension. This input file also benefits from MyMoneyApp encryption. In there, the format is as
follow: \newline \centerline{amount date company expenseType}
\newline\centerline{-1000000 2018-04-08 MyMoneyAppTeam Business}
\newline\newline The input file uses white spaces as delimiter and new line characters as different transaction input (line). MyMoneyApp is designed to ignore white spaces at the beginning of each transaction input within the file input and white spaces at the end before a new line character is found. 
\newline \newline
Other input files are generated as the user registers and logins for the first time. However, those input files are independent of the user and are used as private persistent data storage used within the app. Those other input files are text files (.txt extension) as well and are named as follow: \textbf{user\textunderscore not\textunderscore found.txt} and \textbf{login\textunderscore info.txt}\newline\newline\newline\textbf{user\textunderscore not\textunderscore found.txt} is independent of the user and is only used as a string input to show an error to the user as he enters an incorrect login information. The user is prompted to create an account or enter his correct credentials. This input file is used to change the message MyMoneyApp will show to the user on wrong logging without needing intervention of the development team.\newline\newline\newline\textbf{login\textunderscore info.txt} is independent of the user as well. This persistent data file is used to store login information of the user. login\textunderscore info.txt benefits from MyMoneyApp encryption to ensure that user login data is privately secured from unscrupulous hands.

\section{Description of Output Files}

MyMoneyApp generates text files from the \textbf{transactions\textunderscore data.txt} input file. The output files generated by the application reflect its internal data manipulation algorithm.\newline The MyMoneyApp algorithm analyzes \textbf{transactions\textunderscore data.txt} line by line and models each transaction into a transaction java object. Each transaction object is then stored in a hashtable, where its key is its date, and each key corresponds to a list of transactions that are added as many transactions might have the same date. Then the algorithm is able to organize transactions by date (where it happens to be a key on a hashtable of transactions lists). Finally, the algorithm exports each list with its respective key as a file, where the file name is its key (date) and the content of the list is the content of its respective file. This is achieve to use those files a cache and manipulate them easily as the user is able to change values.
\newline\newline The user is able to test this feature as those files are publicly available. However, internally, MyMoneyApp development team ensures functionality by running unit tests of this algorithm as follows: Insert a new transaction file with one transaction, let the algorithm do its processing, then check if there exists a file with the respective transaction date and its file content matches the new transaction file line input (which is a transaction by itself as well). \newline Users are able to test this by providing their transaction file and seeing if its corresponding transaction file by date is generated. Both files are stored in the same root folder. The user is not able to perform any steps in between, as the algorithm is fully automated and is not expecting any interaction with the user by any means.

\clearpage

\section{References}
Roger S Pressman, Software Engineering: A Practitioner's Approach, 7th edition, McGraw-Hill

\appendix
\clearpage

\listoftables





\end{document}


